\documentclass[onecolumn,11pt,oneside]{article}
\usepackage[latin1]{inputenc}
\usepackage[english]{babel}
\usepackage{a4wide}
\usepackage[round]{natbib}
\usepackage[draft]{fixme}
\usepackage{url}
\usepackage{hyperref}
\usepackage{frcursive}
\usepackage{eurosym}
\usepackage{xspace}
\usepackage{moreverb}
\usepackage{multicol}



% Document
\begin{document}

\section{The notorious vector example}

    It happens fairly frequently, in fact basically all the time, that we need to
    store objects of the same type (say integers) to treat them later on or to
    represent some higher abstraction like a polynomial.

    Moreover in many of these applications, the number of elements to be stored
    is not known in advance (during compile time), but depends on the current
    execution (runtime). Consider multiplying two polynomials: Their maximal degree
    increases and we need more storage for the exponents and coefficients.
    This however depends on the user input and is therefore not known at compile time.

    It is therefore crucial to have a good abstraction of such dynamically sized
    containers, that ``abstract away'' the complexity of allocating and releasing
    memory and are as easy to use as, let's say a Python list.
    There are several containers in the standard library that solve this
    issue, like
        \begin{itemize}
            \item \textit{std::list}
            \item \textit{std::deque}
            \item \textit{std::vector}.
        \end{itemize}
    With each of them having its own pros and cons.

\subsection{Requirements}
    Your implementation must not leak memory and allow access to existing
    elements.
    In this TP we constrain ourselves to integers and we therefore want
    \begin{itemize}
        \item \textbf{constructor:} Initialises an empty vector
        \item \textbf{destructor:} ``Deletes'' the object
        \item \textbf{push\_back:} Appends a given element to the existing vector
        \item \textbf{pop\_back:} Remove the last element from the vector
        \item \textbf{size:} Returns the number of currently stored elements
        \item \textbf{read-access:} Overload the \textit{operator[]} such that it
              takes an index as argument and returns the value at this
              position in a read-only manner.
        \item \textbf{read-write-access:} Overload the \textit{operator[]} such that it
              takes an index as argument and returns the value at this
              position such that it can be read from and written to.
    \end{itemize}
    In addition, the last two functions should throw an error if the user
    wants to access an element that does not exist (``out-of-bounds-error'').

    Once you have correctly implemented all member functions, the program should compile.
    It corresponds to a little benchmark comparing your vector implementation to std::vector.

\subsection*{Reminder build process}
    Here we have a CMake project.
    Assuming you are currently in the folder containing the CMakeLists.txt do
    \begin{enumerate}
        \item mkdir build \&\& cd build
        \item cmake ..
        \item make
    \end{enumerate}

    This should create an executable called ``vectors1''.
\end{document}